%begin_custom_header
\documentclass[12pt]{article}	% RECOMB: "at least 11 point font size on U.S. standard 8 1/2 by 11 inch paper with no less than one inch margin all around."
%\documentclass{nature}
\usepackage[utf8]{inputenc}   % umlauts etc.
\usepackage[english]{babel}
\usepackage [autostyle, english = american]{csquotes}
\MakeOuterQuote{"}
\usepackage[hidelinks]{hyperref}
\setlength\parindent{25pt}
% ----------------------------------
%\usepackage[backend=biber,style=nature,sorting=none,url=true]{biblatex}
% url = false. There are also isbn, doi etc., similar options.
%\addbibresource{/Volumes/DataHub/Publishing/z-misc/zotero-library/better_bibtex_standalone.bib}

% ----------------------------------
% Citation style 	biblatex stylename
% ----------------------------------
% 	ACS				chem-acs
% 	AIP				phys (*)
% 	Natur			nature
% 	Science			science
% 	IEEE			ieee
% 	Chicago			chicago-authordate
% 	MLA				mla
% 	APA				apa
% ----------------------------------
% sorting options:
% ----------------------------------
%	nty 		Sort by name, title, year.
%	nyt 		Sort by name, year, title.
%	nyvt 		Sort by name, year, volume, title.
%	anyt 		Sort by alphabetic label, name, year, title.
%	anyvt 		Sort by alphabetic label, name, year, volume, title.
%	ynt 		Sort by year, name, title.
%	ydnt 		Sort by year (descending), name, title.
%	none 		Do not sort at all. All entries are processed in citation order.
% ----------------------------------
\newcommand{\harpoon}{\overset{\rightharpoonup}}
\newtheorem{theorem}{Theorem}
\usepackage{verbatim} % multiline comment
\usepackage{graphicx}
\graphicspath{{/Volumes/DataHub/Publishing/Figures/}}
\setlength\fboxsep{0pt} % figure border padding
\setlength\fboxrule{1pt} % figure outline
\usepackage[fleqn]{amsmath}  % also \documentclass[fleqn]{article}
\usepackage[margin=1in]{geometry}
%lmargin rmargin tmargin bmargin
	\geometry{
	 a4paper,
	 %total={170mm,257mm},
	 left=20mm
	 ,right=20mm
	 %,top=30mm
	 %,bottom=13mm
	 }


%\abovedisplayskip=0pt
%\abovedisplayshortskip=0pt
%\belowdisplayskip=0pt
%\belowdisplayshortskip=0pt
%
%\setlength{\mathindent}{0pt}
%
\usepackage{amsfonts} % for R (real numbers)
\usepackage{float}
% \usepackage[font=scriptsize,labelfont=bf]{caption}
\usepackage{booktabs, cellspace, hhline} % \cellspacetoplim; http://tex.stackexchange.com/questions/302960/modify-arraystretch-for-a-single-row-in-table


\usepackage[percent]{overpic}
\usepackage[export]{adjustbox}
% ----------------------------------
%Squeezing the Vertical White Space
%http://www.terminally-incoherent.com/blog/2007/09/19/latex-squeezing-the-vertical-white-space/
% 	THIS FIXES THE PROBLEM OF SUBSECTIONS STARTING IN A NEW PAGE
%   \setlength{\parskip}{5pt}
%   \setlength{\parsep}{10pt}
%   \setlength{\headsep}{0pt}
%   \setlength{\topskip}{0pt}
%   \setlength{\topmargin}{0pt}
%   \setlength{\topsep}{0pt}
%   \setlength{\partopsep}{10pt}
\usepackage[compact]{titlesec}
\titlespacing{\section}{0pt}{*2}{*2} % {left margin} {above-skip} {below-kip} , The * notation replaces the formal notation using plus/minus and etc.
\titlespacing{\subsection}{0pt}{*1}{*1}
\titlespacing{\subsubsection}{0pt}{*1}{*1}
% ----------------------------------
%\newenvironment{absolutelynopagebreak}
%  {\par\nobreak\vfil\penalty0\vfilneg
%   \vtop\bgroup}
%  {\par\xdef\tpd{\the\prevdepth}\egroup
%   \prevdepth=\tpd}
% ----------------------------------
\newcommand{\bfl}{\begin{flushleft}}
\newcommand{\efl}{\end{flushleft}}
\newcommand{\mys }{\hspace{0.1cm}}
\newcommand{\figfont}{\footnotesize}

\usepackage[table]{xcolor} % for \arrayrulecolor{yellow}, changes hline and vline colors
\definecolor{myTableLines}{HTML}{0a84f7}

\usepackage{relsize} % \tiny, \scriptsize, \footnotesize, \small, \normalsize, \large, \Large, \LARGE, \huge, \Huge
                    % if the current font size is 'normalsize', then using \relsize{-1} ==> go down one level ==> font = small
                    % and using \relsize{1}  ==> font = large
\newcommand{\myC}[1]{{\relsize{-1}{$\mathcal{#1}$}}} % as in 'my complexity class'

% padding in tables
\newcommand{\myPadTop}{6.5pt}
\newcommand{\myPadBottom}{4pt}

\mathchardef\myhyphen="2D
\usepackage{amssymb}%for QED symbol \blacksquare
%end_custom_header

%begin_custom_header
%end_custom_header
\begin{document}
%begin_custom_content
\section{Applications and Extensions of the NEP Model}\label{sup_extensions}
		\subsection{Applications:}
         An immediate application of the model is to complement statistical tests used to infer the quality and coverage of large-scale interactome-mapping wet experiments  \cite{rolland_proteome-scale_2014} or \textit{in-silico} network inference  \cite{mitra_integrative_2013}, by testing whether the resulting networks over- or under-represents real interactions relative to the prediction. But there is potential for insights into functional aspects of biological systems. For example, while the model was semantically interpreted Section \ref{sup_semantics} (summerized in Table \ref{informal_table}) in the context of evolution, it can also be interpreted in the context of regulation (i.e. "up-regulate"/"down-regulate" rather than "conserve"/"delete"). Through this interpretation, a new way of approaching cancer for example is possible: what system-wide alterations to a regulatory network of a healthy cell would result in the hardest optimization task to restore the total number of damaging regulatory interactions to a certain thereshold? This can shed lights on the "intractable" regulatory perturbations that Nature's algorithm (random-variation/non-random selection) has not managed to proof against. Such approach can complement correlation-based studies  \cite{colquhoun_investigation_2014} which (even when statistically sound) do not necessarily reveal underlying causations.  The applicability of the NEP to regulatory network is accomplished by using a matrix Oracle that advices on edges rather than a sequence Oracle that advices on genes (as is the case in the main text). This variant of the model is discussed in details in  \cite{atiia_computational_2017}, but we include a summary of it here. A biological network of $n$ genes $(g_1, g_2,\dots, g_n)$ can be represent as an adjacency matrix $M=\big [m_{jk}\big ]$, $1\leq j,k\leq n$ where $m_{jk} = +1, -1,\text{or } 0$ implies, respectively, that $g_j$ promotes, inhibits, or doesn't interact with $g_k$. At a given point in evolutionary time, some interactions may become damaging to the overall fitness of the organism: $g_j$ promotes (inhibits) $g_k$ when the latter should in fact be inhibited(promoted). Conversely, some interactions can become advantageously essential: $g_j$ promotes (inhibits) $g_k$ when the latter should indeed be promoted (inhibited). Let $A=\big [a_{jk}\big ]$ represent a hypothetical "ideal" regulatory state, such that $a_{jk}\in\{+1,-1\}$ if $m_{jk}\neq 0$ and $a_{jk}=0$ otherwise. We refer to $A$ as an "Oracle advice" (OA) on the network. While $m_{jk}\neq 0$ describes what the effect of $g_j$ on $g_k$ actually is, $a_{jk}$ describes what that effect should \textit{ideally} be. A beneficial (damaging) interaction is one where $m_{jk} \times a_{jk} = 1 $ $(m_{jk}\times a_{jk}=-1)$. In other words, an interaction is beneficial (damaging) if it is in agreement (disagreement) with what the Oracle says that interaction should ideally be. Assume for example that $g_j$ promotes $g_k$, i.e. $m_{jk}=+1$, but the OA says that interaction should ideally be inhibitory instead, i.e. $a_{jk}=-1$, then $m_{jk}\times a_{jk}=-1$ implies the real disagrees with the ideal and the interaction is deemed damaging. 

        The benefit (damage) score of each gene $g_j$, given an OA, is the sum of beneficial (damaging) interactions that $g_j$ is \textit{projecting} onto (out-edges) or \textit{attracting} from (in-edges) other genes. More precisely, the benefit score of $g_j$ is defined as:

		$b_j = \sum\limits_{k=1}^{n} m_{jk} \oplus a_{jk} \hspace{0.1cm}+\hspace{0.1cm} \sum\limits_{k=1}^{n} m_{kj} \oplus a_{kj} \quad\textrm{where:}$

		$\hspace{2cm}m_{xy} \oplus a_{xy}  =	\scriptscriptstyle{\begin{cases}	% in math mode, use scriptstyle/scriptscriptstyle, not small/tiny
											1 & \quad\textrm{if}\quad m_{xy} \times a_{xy} >0 \\
											0 & \quad\textrm{otherwise}
									\end{cases}
									}$

		and similarly the damage score is:

		$d_j = \sum\limits_{k=1}^{n} m_{jk} \ominus a_{jk} \hspace{0.1cm}+\hspace{0.1cm} \sum\limits_{k=1}^{n} m_{kj} \ominus a_{kj} \quad\textrm{where:}$

		$\hspace{2cm}m_{xy} \ominus a_{xy}  = \scriptscriptstyle{\begin{cases}	% in math mode, use scriptstyle/scriptscriptstyle, not small/tiny
											1 & \quad\textrm{if}\quad m_{xy} \times a_{xy} < 0 \\
											0 & \quad\textrm{otherwise}
									\end{cases}
									}$

		An organism is clearly better off conserving a gene $g_j$ if its benefit $b_j\neq 0$  and damage $d_j=0$, and deleting $g_j$ if $d_j\neq 0$ and $b_j=0$. We refer to such genes as \textit{unambiguous}. Clearly a degree-1 leaf gene $g_k$ (i.e. it is involved in a single interaction with another gene) is always unambiguous. A degree-2 $g_k$ can have one of four possible ($b_k$,$d_k$) values: 00, 01, 10, 11 with each digit representing an interaction (edge) and 0 or 1 implying the interaction is beneficial or damaging, respectively, and as such $g_k$ has a 50\% chance of being unambiguous under a random OA (i.e. equal likelihood of an interaction being deemed beneficial or damaging by the Oracle). As the degree $d$ of $g_k$ increases linearly, the probability of it being unambiguous under some OA decreases exponentially (namely,  $prob. = 2^{1-d}$). The network evolution problem (NEP) is that of defining the following function $f$:

		\noindent {\scriptsize $f:\boldsymbol{G}  \rightarrow \{0,1\} \mys \textrm{maximizing} \mys  \sum\limits_{j=1}^{n} f(g_j)\times b_j \mys\mys\textrm{s.t.} \mys\mys \Bigg(  \sum\limits_{j=1}^{n} f(g_j)\times d_j  \Bigg)  \leq \boldsymbol{t} $}

        \subsection{Extensions:}
		There are aspects of the model that can be extended. In this work we treated all interactions as equal, but in reality some interactions are more potent than others. Future work could extend the model by considering the potency of each interaction, a not so trivial task since no large-scale data exist yet to facilitate its inference. The model is also static, in the sense that assigning benefit/damage to a gene is based on immediate neighbours only. The implication is that all genes are equal, but in reality a central gene (many shortest paths pass through it) has much more effect network-wide than a gene residing at the periphery of the network. Trivially, a dynamic variant is where the cascading effect of a gene's beneficial (damaging) effect does not change the complexity class of NEP, although its simulations will be more computationally demanding.

\clearpage %https://tex.stackexchange.com/questions/2958/why-is-newpage-ignored-sometimes
\printbibliography
\end{document}
%end_custom_content
