%begin_custom_header
\documentclass[12pt]{article}	% RECOMB: "at least 11 point font size on U.S. standard 8 1/2 by 11 inch paper with no less than one inch margin all around."
%\documentclass{nature}
\usepackage[utf8]{inputenc}   % umlauts etc.
\usepackage[english]{babel}
\usepackage [autostyle, english = american]{csquotes}
\MakeOuterQuote{"}
\usepackage[hidelinks]{hyperref}
\setlength\parindent{25pt}
% ----------------------------------
%\usepackage[backend=biber,style=nature,sorting=none,url=true]{biblatex}
% url = false. There are also isbn, doi etc., similar options.
%\addbibresource{/Volumes/DataHub/Publishing/z-misc/zotero-library/better_bibtex_standalone.bib}

% ----------------------------------
% Citation style 	biblatex stylename
% ----------------------------------
% 	ACS				chem-acs
% 	AIP				phys (*)
% 	Natur			nature
% 	Science			science
% 	IEEE			ieee
% 	Chicago			chicago-authordate
% 	MLA				mla
% 	APA				apa
% ----------------------------------
% sorting options:
% ----------------------------------
%	nty 		Sort by name, title, year.
%	nyt 		Sort by name, year, title.
%	nyvt 		Sort by name, year, volume, title.
%	anyt 		Sort by alphabetic label, name, year, title.
%	anyvt 		Sort by alphabetic label, name, year, volume, title.
%	ynt 		Sort by year, name, title.
%	ydnt 		Sort by year (descending), name, title.
%	none 		Do not sort at all. All entries are processed in citation order.
% ----------------------------------
\newcommand{\harpoon}{\overset{\rightharpoonup}}
\newtheorem{theorem}{Theorem}
\usepackage{verbatim} % multiline comment
\usepackage{graphicx}
\graphicspath{{/Volumes/DataHub/Publishing/Figures/}}
\setlength\fboxsep{0pt} % figure border padding
\setlength\fboxrule{1pt} % figure outline
\usepackage[fleqn]{amsmath}  % also \documentclass[fleqn]{article}
\usepackage[margin=1in]{geometry}
%lmargin rmargin tmargin bmargin
	\geometry{
	 a4paper,
	 %total={170mm,257mm},
	 left=20mm
	 ,right=20mm
	 %,top=30mm
	 %,bottom=13mm
	 }


%\abovedisplayskip=0pt
%\abovedisplayshortskip=0pt
%\belowdisplayskip=0pt
%\belowdisplayshortskip=0pt
%
%\setlength{\mathindent}{0pt}
%
\usepackage{amsfonts} % for R (real numbers)
\usepackage{float}
% \usepackage[font=scriptsize,labelfont=bf]{caption}
\usepackage{booktabs, cellspace, hhline} % \cellspacetoplim; http://tex.stackexchange.com/questions/302960/modify-arraystretch-for-a-single-row-in-table


\usepackage[percent]{overpic}
\usepackage[export]{adjustbox}
% ----------------------------------
%Squeezing the Vertical White Space
%http://www.terminally-incoherent.com/blog/2007/09/19/latex-squeezing-the-vertical-white-space/
% 	THIS FIXES THE PROBLEM OF SUBSECTIONS STARTING IN A NEW PAGE
%   \setlength{\parskip}{5pt}
%   \setlength{\parsep}{10pt}
%   \setlength{\headsep}{0pt}
%   \setlength{\topskip}{0pt}
%   \setlength{\topmargin}{0pt}
%   \setlength{\topsep}{0pt}
%   \setlength{\partopsep}{10pt}
\usepackage[compact]{titlesec}
\titlespacing{\section}{0pt}{*2}{*2} % {left margin} {above-skip} {below-kip} , The * notation replaces the formal notation using plus/minus and etc.
\titlespacing{\subsection}{0pt}{*1}{*1}
\titlespacing{\subsubsection}{0pt}{*1}{*1}
% ----------------------------------
%\newenvironment{absolutelynopagebreak}
%  {\par\nobreak\vfil\penalty0\vfilneg
%   \vtop\bgroup}
%  {\par\xdef\tpd{\the\prevdepth}\egroup
%   \prevdepth=\tpd}
% ----------------------------------
\newcommand{\bfl}{\begin{flushleft}}
\newcommand{\efl}{\end{flushleft}}
\newcommand{\mys }{\hspace{0.1cm}}
\newcommand{\figfont}{\footnotesize}

\usepackage[table]{xcolor} % for \arrayrulecolor{yellow}, changes hline and vline colors
\definecolor{myTableLines}{HTML}{0a84f7}

\usepackage{relsize} % \tiny, \scriptsize, \footnotesize, \small, \normalsize, \large, \Large, \LARGE, \huge, \Huge
                    % if the current font size is 'normalsize', then using \relsize{-1} ==> go down one level ==> font = small
                    % and using \relsize{1}  ==> font = large
\newcommand{\myC}[1]{{\relsize{-1}{$\mathcal{#1}$}}} % as in 'my complexity class'

% padding in tables
\newcommand{\myPadTop}{6.5pt}
\newcommand{\myPadBottom}{4pt}

\mathchardef\myhyphen="2D
\usepackage{amssymb}%for QED symbol \blacksquare
%end_custom_header

%begin_custom_header
%end_custom_header
\begin{document}
%begin_custom_content
\newpage
\section{Formal Definition of the Network Evolution Problem (NEP)} \label{sup_NEP_definition}
				\noindent Given: %http://tex.stackexchange.com/questions/82240/unwanted-space-before-flalign
				\vspace{.3cm}
				{\small
				%\setlength{\abovedisplayskip}{0pt}\setlength{\belowdisplayskip}{0pt}\setlength{\abovedisplayshortskip}{0pt}\setlength{\belowdisplayshortskip}{0pt}
						\begin{flalign*} % {align} produces equation numbers
							&\mys\mys\boldsymbol{G}  = (g_1,g_2,\dots,g_n) \textrm{, }  \boldsymbol {A}  = (a_1,a_2,\dots,a_n)  \textrm{, }  a_j \in \{+1,0,-1\} \textrm{, } \mys\boldsymbol {t}\in\mathbb{R}\textrm{, \mys and }  \\
							&\mys\mys\boldsymbol{M}  = \big [m_{jk}\big ] \quad \text{where} \quad  m_{jk} \in  \mathbb{R}, \mys \quad\forall j,k, \mys 1\leq j,k \leq n %\mathbb{R},
											%\begin{bmatrix}%nside math mode, in order to use a different (smaller) font, you could/should use \scriptstyle or \scriptscriptstyle
											%				m_{11} & m_{12} & \dots  & m_{1n} \\
											%				m_{21} & m_{22} & \dots  & m_{2n} \\
											%				\vdots & \vdots & \ddots & \vdots \\
											%				m_{n1} & m_{n2} & \dots & m_{nn}
											%			\end{bmatrix}
											%	\quad \text{where} \quad  m_{jk} \in  \{+1,0,-1\} & %\mathbb{R},
						\end{flalign*}
				}

				\noindent Let:
				{\small
				%\setlength{\abovedisplayskip}{0pt}\setlength{\belowdisplayskip}{0pt}\setlength{\abovedisplayshortskip}{0pt}\setlength{\belowdisplayshortskip}{0pt}
						\begin{flalign*}
							 \mys\mys\boldsymbol {B}  = (b_1,b_2,\dots, b_n) \textrm{,}\hspace{0.2cm}&\textrm{where}\hspace{0.2cm} b_j = \sum\limits_{k=1}^{n} m_{jk} \oplus a_k \hspace{0.15cm}+\hspace{0.15cm}\sum\limits_{k=1}^{n} m_{kj} \oplus a_j \hspace{.26cm}\textrm{and} \hspace{.4cm} &\\
								 &  m_{xy} \oplus a_y  =
											\scriptscriptstyle{\begin{cases}	% in math mode, use scriptstyle/scriptscriptstyle, not small/tiny
													|m_{xy}| & \mys\textrm{if}\quad m_{xy} \times a_y >0 \\
													\mys\mys 0 & \mys\textrm{otherwise}
											\end{cases}
											} &
						\end{flalign*}
				}
				{\small
						%\setlength{\abovedisplayskip}{0pt}\setlength{\belowdisplayskip}{0pt}\setlength{\abovedisplayshortskip}{0pt}\setlength{\belowdisplayshortskip}{0pt}
						\begin{flalign*}
						\mys\mys\boldsymbol {D}  = (d_1,d_2,\dots, d_n) \textrm{,}\hspace{0.15cm}&\textrm{where}\hspace{0.2cm} d_j = \sum\limits_{k=1}^{n} m_{jk} \ominus a_k \hspace{0.15cm}+\hspace{0.15cm} \sum\limits_{k=1}^{n} m_{kj}\ominus a_j \hspace{.21cm}\textrm{and} \hspace{.4cm} &\\
								 & m_{xy} \ominus a_y =
											\scriptscriptstyle{\begin{cases}
															|m_{xy}| & \mys\textrm{if}\quad m_{xy} \times a_y < 0 \\
															\mys\mys 0 & \mys\textrm{otherwise}
											\end{cases}
											} &
						\end{flalign*}
					}

				\noindent Define:
					{\small
					  %\setlength{\abovedisplayskip}{0pt}\setlength{\belowdisplayskip}{0pt}\setlength{\abovedisplayshortskip}{0pt}\setlength{\belowdisplayshortskip}{0pt}
						\begin{flalign*}
							&\mys\mys f:\boldsymbol{G}  \rightarrow \{0,1\} \mys \textrm{maximizing} \mys  \sum\limits_{j=1}^{n} f(g_j)\times b_j
							\mys\mys\textrm{s.t.} \mys\mys \Bigg(  \sum\limits_{j=1}^{n} f(g_j)\times d_j  \Bigg)  \leq \boldsymbol{t} &
						\end{flalign*}
					}

Table \ref{informal_table} provides a summary of each element of NEP and its corresponding semantic interpretation in biological context (see also the main text for more on the semantics of NEP in biological context).

\begin{comment}
	The network can equivalently be represented as an adjacency matrix $M$, whereby a non-zero entry $m_{jk}$ indicates the existence of an interaction between genes $g_j$ and $g_k$
	in which the latter is the target of the former.
	The sign of a non-zero entry in $M$ indicates whether $g_j$'s effect on its target $g_k$ is promotional or inhibitory in nature, indicated with
	$+1$ or $-1$, respectively.
	A hypothetical Oracle advice (OA) on all or some of the genes simulates the evolutionary pressure on the
	network, and
	is represented as a ternary sequence $A = (a_1,a_2,\dots,a_n)$ where: $a_j=+1$ ($a_j=-1$) implies
	the organism would be better off conserving (deleting) $g_j$; $a_j=0$ implies the Oracle has no opinion on $g_j$.
	While $m_{jk}$ describes what the effect of $g_j$ on $g_k$ actually \textit{is}, $a_k$ describes whether that effect \textit{should} ideally be.
	An interaction $m_{jk}$
	is beneficial if it is in agreement with what the Oracle
	says $g_k$ should be (i.e. either $(m_{jk}=+1$ AND $a_k=+1)$ OR $(m_{jk}=-1$ AND $a_k=-1)$), and damaging if it is
	in disagreement with what the Oracle says $g_k$ should be (i.e. either $(m_{jk}=+1$ AND $a_k=-1)$ OR $(m_{jk}=-1$ AND $a_k=+1)$).
	Each gene $g_j$ is henceforth assigned a benefit (damage) score $b_j$ ($d_j$) depending on how many beneficial (damaging)
	interactions it \textit{projects} onto or \textit{attracts} from other genes through its
	outgoing and incoming edges, respectively.
	Each beneficial (damaging) interaction therefore adds $|m_{jk}|$ to the benefit (damage) score of both the source gene $g_j$ and the target gene $g_k$.
	A gene can therefore have both non-zero benefit and damage score under a given pressure scenario, and so the optimization problem is:
	what subset of genes should be conserved and which should be deleted (=define $f$) so as to maximize (minimize) the number of
	interactions that are in agreement (disagreement) with the OA?
	The OA can be imposed by conserving (deleting) every gene $\boldsymbol{g_j}$ where $a_j = +1$ ($a_j=-1$).
	However,
	conserving $\boldsymbol{g_j}$ can inadvertently contribute to a violation of  the OA if
	$g_j$ happens to be a promoter (inhibitor) of one or more $g_k$ where $a_k=-1$($a_k=+1$), and
	deleting $\boldsymbol{g_j}$ can inadvertently contribute to a violation of the OA if $g_j$
	happens to be a promoter (repressor) of one or more $g_k$ where $a_k=+1$ ($a_k=-1$).
	The idealistic pursuit of enforcing an OA is complicated by the reality of network connectivity.
\end{comment}
%end_custom_content
\printbibliography
\end{document}
