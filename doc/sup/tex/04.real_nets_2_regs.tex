%begin_custom_header
\documentclass[12pt]{article}	% RECOMB: "at least 11 point font size on U.S. standard 8 1/2 by 11 inch paper with no less than one inch margin all around."
%\documentclass{nature}
\usepackage[utf8]{inputenc}   % umlauts etc.
\usepackage[english]{babel}
\usepackage [autostyle, english = american]{csquotes}
\MakeOuterQuote{"}
\usepackage[hidelinks]{hyperref}
\setlength\parindent{25pt}
% ----------------------------------
%\usepackage[backend=biber,style=nature,sorting=none,url=true]{biblatex}
% url = false. There are also isbn, doi etc., similar options.
%\addbibresource{/Volumes/DataHub/Publishing/z-misc/zotero-library/better_bibtex_standalone.bib}

% ----------------------------------
% Citation style 	biblatex stylename
% ----------------------------------
% 	ACS				chem-acs
% 	AIP				phys (*)
% 	Natur			nature
% 	Science			science
% 	IEEE			ieee
% 	Chicago			chicago-authordate
% 	MLA				mla
% 	APA				apa
% ----------------------------------
% sorting options:
% ----------------------------------
%	nty 		Sort by name, title, year.
%	nyt 		Sort by name, year, title.
%	nyvt 		Sort by name, year, volume, title.
%	anyt 		Sort by alphabetic label, name, year, title.
%	anyvt 		Sort by alphabetic label, name, year, volume, title.
%	ynt 		Sort by year, name, title.
%	ydnt 		Sort by year (descending), name, title.
%	none 		Do not sort at all. All entries are processed in citation order.
% ----------------------------------
\newcommand{\harpoon}{\overset{\rightharpoonup}}
\newtheorem{theorem}{Theorem}
\usepackage{verbatim} % multiline comment
\usepackage{graphicx}
\graphicspath{{/Volumes/DataHub/Publishing/Figures/}}
\setlength\fboxsep{0pt} % figure border padding
\setlength\fboxrule{1pt} % figure outline
\usepackage[fleqn]{amsmath}  % also \documentclass[fleqn]{article}
\usepackage[margin=1in]{geometry}
%lmargin rmargin tmargin bmargin
	\geometry{
	 a4paper,
	 %total={170mm,257mm},
	 left=20mm
	 ,right=20mm
	 %,top=30mm
	 %,bottom=13mm
	 }


%\abovedisplayskip=0pt
%\abovedisplayshortskip=0pt
%\belowdisplayskip=0pt
%\belowdisplayshortskip=0pt
%
%\setlength{\mathindent}{0pt}
%
\usepackage{amsfonts} % for R (real numbers)
\usepackage{float}
% \usepackage[font=scriptsize,labelfont=bf]{caption}
\usepackage{booktabs, cellspace, hhline} % \cellspacetoplim; http://tex.stackexchange.com/questions/302960/modify-arraystretch-for-a-single-row-in-table


\usepackage[percent]{overpic}
\usepackage[export]{adjustbox}
% ----------------------------------
%Squeezing the Vertical White Space
%http://www.terminally-incoherent.com/blog/2007/09/19/latex-squeezing-the-vertical-white-space/
% 	THIS FIXES THE PROBLEM OF SUBSECTIONS STARTING IN A NEW PAGE
%   \setlength{\parskip}{5pt}
%   \setlength{\parsep}{10pt}
%   \setlength{\headsep}{0pt}
%   \setlength{\topskip}{0pt}
%   \setlength{\topmargin}{0pt}
%   \setlength{\topsep}{0pt}
%   \setlength{\partopsep}{10pt}
\usepackage[compact]{titlesec}
\titlespacing{\section}{0pt}{*2}{*2} % {left margin} {above-skip} {below-kip} , The * notation replaces the formal notation using plus/minus and etc.
\titlespacing{\subsection}{0pt}{*1}{*1}
\titlespacing{\subsubsection}{0pt}{*1}{*1}
% ----------------------------------
%\newenvironment{absolutelynopagebreak}
%  {\par\nobreak\vfil\penalty0\vfilneg
%   \vtop\bgroup}
%  {\par\xdef\tpd{\the\prevdepth}\egroup
%   \prevdepth=\tpd}
% ----------------------------------
\newcommand{\bfl}{\begin{flushleft}}
\newcommand{\efl}{\end{flushleft}}
\newcommand{\mys }{\hspace{0.1cm}}
\newcommand{\figfont}{\footnotesize}

\usepackage[table]{xcolor} % for \arrayrulecolor{yellow}, changes hline and vline colors
\definecolor{myTableLines}{HTML}{0a84f7}

\usepackage{relsize} % \tiny, \scriptsize, \footnotesize, \small, \normalsize, \large, \Large, \LARGE, \huge, \Huge
                    % if the current font size is 'normalsize', then using \relsize{-1} ==> go down one level ==> font = small
                    % and using \relsize{1}  ==> font = large
\newcommand{\myC}[1]{{\relsize{-1}{$\mathcal{#1}$}}} % as in 'my complexity class'

% padding in tables
\newcommand{\myPadTop}{6.5pt}
\newcommand{\myPadBottom}{4pt}

\mathchardef\myhyphen="2D
\usepackage{amssymb}%for QED symbol \blacksquare
%end_custom_header

%begin_custom_header
%end_custom_header
\begin{document}
%begin_custom_content
\newpage
\subsection{Regulatory networks}\label{sup_realnets_reg}
    Regulatory networks (details and references in Table \ref{tab:networks_summary_Reg}, raw data and source code available in  \cite{atiia_case-study_2017}) are all directed, with some being partially signed (RegulonDB and TRRUST). The nodes in regulatory networks can be transcription factors, genes (which can refer to the protein or mRNA), or small RNAs. All networks originally contain exclusively experimentally-validated interactions, with the exception of Liu and RegulonDB which contain computationally (\textit{in-silico}) inferred interactions which were excluded. In the case of RegulonDB, only interactions with  `strong' or `confirmed' experimental evidence are included, and since none of the interactions involving small RNAs had such evidence, they were eliminated. The remaining interactions were therefore exclusively between transcription factors. In miRTarBase networks, only interactions with strong experimental evidence (elucidated through reporter assays or western blot experiments) are included. Furthermore, interactions where the species of source and target genes are different were excluded (presumably, these original from  transgenic studies).

    The ENCODE proximal network is an overall consolidated network of transcriptional interactions in humans, with some interactions being obtained by further consolidation with PPI network (detailed in supplementary materials of  \cite{gerstein_architecture_2012}). The other two ENCODE networks on the other hand are generated from specific human cell lines (GM and K562). The TRRUST network is unique in that it was obtained by data mining ${\sim}$20 million literature abstracts from  Medline (2014), out of which ${\sim}$23K sentences were nominated to contain potential descriptions of regulatory interactions  \cite{han_trrust:_2015}. These sentences underwent successive rounds of manual inspections. TRRUST network also includes information about the nature of interactions and the number of studies supporting it. For interactions deemed promotional by some studies and inhibitory by others, we picked the sign randomly by flipping a crooked coin proportional to the number of studies that support one type or another (for example, if 3 studies report an interaction as `promotional' and 1 reports it as `inhibitory', we would consider the interaction to be `promotional' with 75\% likelihood). TRRUST authors aimed to create a high-quality network that can serve as a gold-standard to other large-scale studies aiming to map transcriptome interactions in humans. The same crooked coin strategy was used in RegulonDB network. Figure SI \ref{fig:deg_dist_Reg} shows the degree distributions of regulatory networks and their corresponding synthetic analogs. Despite the diverse methods that were behind the mapping of these networks (in contrast to PPIs, where Y2H method is dominant), the mLmH property still holds with lower-degree nodes in particularly being of almost the same frequency in the majority of networks.
    \newpage
    \begin{table}[H]%[!htb]
        \centering
        \setlength\arrayrulewidth{.1pt}\arrayrulecolor{myTableLines}%\arrayrulecolor[HTML]{0a84f7}
        \scriptsize
            \setlength\cellspacetoplimit{\myPadTop}\setlength\cellspacebottomlimit{\myPadBottom}
            \begin{tabular}{@{}Sc Sc|Sc|Sc|Sc|Sc|Sc@{}}
                \cline{2-7}
                    & \textbf{\normalsize Regulatory Network} & \textbf{\normalsize no. nodes} & \textbf{\normalsize no. edges}	& \textbf{\normalsize e2n ratio} & \textbf{\normalsize directed? } & \textbf{\normalsize signed? }%&  \textbf{\normalsize n2e ratio}
                    %\\[.05cm] \cline{1-6}
    				%    \multirow{10}{*}{\textbf{\normalsize Regulatory}} & TRRUST  \cite{han_trrust:_2015}   &  2718  &  8015  &  2.95  &   0.340
                            \\[.05cm] \cline{2-7}
                                    & Bacteria RegulonDB  \cite{gama-castro_regulondb_2016} & 898       & 1481      & 1.649    & yes & no % & 0.606
                            \\[.05cm] \cline{2-7}
                                    &  ENCODE Proximal  \cite{gerstein_architecture_2012}   & 9057      & 26070     & 2.878    & yes & no %& 0.347
                            \\[.05cm] \cline{2-7}
                                    &  ENCODE K562  \cite{gerstein_architecture_2012}       & 3947      & 9595      & 2.431    & yes & no % & 0.411
                            \\[.05cm] \cline{2-7}
                                    &  ENCODE GM  \cite{gerstein_architecture_2012}         & 3989      & 6971      & 1.748    & yes & no % & 0.572
                            \\[.05cm] \cline{2-7}
                                    & Human Liu  \cite{liu_regnetwork:_2015}                & 3502      & 9606      & 2.743    & yes & no %& 0.365
                            \\[.05cm] \cline{2-7}
                                    & Human TRRUST  \cite{han_trrust:_2015}                 & 2718      & 8015      & 2.949    & yes & yes %& 0.339
                            \\[.05cm] \cline{2-7}
                                    & Human miRTarBase  \cite{chou_mirtarbase_2016}         & 2583      & 5450      & 2.11     & yes & no %& 0.474
                            \\[.05cm] \cline{2-7}
                                    & Mouse Liu  \cite{liu_regnetwork:_2015}                & 1436      & 3673      & 2.558    & yes & no % & 0.391
                            \\[.05cm] \cline{2-7}
                                    & Mouse miRTarBase  \cite{chou_mirtarbase_2016}         & 741       & 1019      & 1.375    & yes & no %& 0.727
                            \\[.05cm] \cline{2-7}
                            %\\[.05cm] \cline{1-6}
            \end{tabular}
            \caption[Summary of regulatory networks.]
                    {
                        Summary of regulatory networks. The direction and sign of an interaction were assigned at random (coin flip) in undirected and/or unsigned networks. References, data and source code publicly available in  \cite{atiia_case-study_2017}.
                    }
            \label{tab:networks_summary_Reg}
        \end{table}


        \begin{figure}[H]%[!htb]
            \includegraphics[width=.49\textwidth]{02.degree-dist/Thesis/Regulatory/Reg.png}
            \includegraphics[width=.49\textwidth]{02.degree-dist/Thesis/Regulatory/NH.png}
            \\
            \includegraphics[width=.49\textwidth]{02.degree-dist/Thesis/Regulatory/NL.png}
            \includegraphics[width=.49\textwidth]{02.degree-dist/Thesis/Regulatory/RN.png}
            \caption
                    [
                        Degree distribution of regulatory networks.
                    ]
                    {
                        Degree distribution of regulatory networks and their corresponding synthetic analogs: no-hubs (NH), no-leaves (NL) and random (RN).
                    }
            \label{fig:deg_dist_Reg}
        \end{figure}

%end_custom_content
\printbibliography
\end{document}
