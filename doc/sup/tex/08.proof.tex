%begin_custom_header
\documentclass[12pt]{article}	% RECOMB: "at least 11 point font size on U.S. standard 8 1/2 by 11 inch paper with no less than one inch margin all around."
%\documentclass{nature}
\usepackage[utf8]{inputenc}   % umlauts etc.
\usepackage[english]{babel}
\usepackage [autostyle, english = american]{csquotes}
\MakeOuterQuote{"}
\usepackage[hidelinks]{hyperref}
\setlength\parindent{25pt}
% ----------------------------------
%\usepackage[backend=biber,style=nature,sorting=none,url=true]{biblatex}
% url = false. There are also isbn, doi etc., similar options.
%\addbibresource{/Users/unbaxabl/Downloads/Publishing/z-misc/zotero-library/better_bibtex_standalone.bib}

% ----------------------------------
% Citation style 	biblatex stylename
% ----------------------------------
% 	ACS				chem-acs
% 	AIP				phys (*)
% 	Natur			nature
% 	Science			science
% 	IEEE			ieee
% 	Chicago			chicago-authordate
% 	MLA				mla
% 	APA				apa
% ----------------------------------
% sorting options:
% ----------------------------------
%	nty 		Sort by name, title, year.
%	nyt 		Sort by name, year, title.
%	nyvt 		Sort by name, year, volume, title.
%	anyt 		Sort by alphabetic label, name, year, title.
%	anyvt 		Sort by alphabetic label, name, year, volume, title.
%	ynt 		Sort by year, name, title.
%	ydnt 		Sort by year (descending), name, title.
%	none 		Do not sort at all. All entries are processed in citation order.
% ----------------------------------
\newcommand{\harpoon}{\overset{\rightharpoonup}}
\newtheorem{theorem}{Theorem}
\usepackage{verbatim} % multiline comment
\usepackage{graphicx}
\graphicspath{{/Users/unbaxabl/Downloads/Publishing/Figures/}}
\setlength\fboxsep{0pt} % figure border padding
\setlength\fboxrule{1pt} % figure outline
\usepackage[fleqn]{amsmath}  % also \documentclass[fleqn]{article}
\usepackage[margin=1in]{geometry}
%lmargin rmargin tmargin bmargin
	\geometry{
	 a4paper,
	 %total={170mm,257mm},
	 left=20mm
	 ,right=20mm
	 %,top=30mm
	 %,bottom=13mm
	 }


%\abovedisplayskip=0pt
%\abovedisplayshortskip=0pt
%\belowdisplayskip=0pt
%\belowdisplayshortskip=0pt
%
%\setlength{\mathindent}{0pt}
%
\usepackage{amsfonts} % for R (real numbers)
\usepackage{float}
% \usepackage[font=scriptsize,labelfont=bf]{caption}
\usepackage{booktabs, cellspace, hhline} % \cellspacetoplim; http://tex.stackexchange.com/questions/302960/modify-arraystretch-for-a-single-row-in-table


\usepackage[percent]{overpic}
\usepackage[export]{adjustbox}
% ----------------------------------
%Squeezing the Vertical White Space
%http://www.terminally-incoherent.com/blog/2007/09/19/latex-squeezing-the-vertical-white-space/
% 	THIS FIXES THE PROBLEM OF SUBSECTIONS STARTING IN A NEW PAGE
%   \setlength{\parskip}{5pt}
%   \setlength{\parsep}{10pt}
%   \setlength{\headsep}{0pt}
%   \setlength{\topskip}{0pt}
%   \setlength{\topmargin}{0pt}
%   \setlength{\topsep}{0pt}
%   \setlength{\partopsep}{10pt}
\usepackage[compact]{titlesec}
\titlespacing{\section}{0pt}{*2}{*2} % {left margin} {above-skip} {below-kip} , The * notation replaces the formal notation using plus/minus and etc.
\titlespacing{\subsection}{0pt}{*1}{*1}
\titlespacing{\subsubsection}{0pt}{*1}{*1}
% ----------------------------------
%\newenvironment{absolutelynopagebreak}
%  {\par\nobreak\vfil\penalty0\vfilneg
%   \vtop\bgroup}
%  {\par\xdef\tpd{\the\prevdepth}\egroup
%   \prevdepth=\tpd}
% ----------------------------------
\newcommand{\bfl}{\begin{flushleft}}
\newcommand{\efl}{\end{flushleft}}
\newcommand{\mys }{\hspace{0.1cm}}
\newcommand{\figfont}{\footnotesize}

\usepackage[table]{xcolor} % for \arrayrulecolor{yellow}, changes hline and vline colors
\definecolor{myTableLines}{HTML}{0a84f7}

\usepackage{relsize} % \tiny, \scriptsize, \footnotesize, \small, \normalsize, \large, \Large, \LARGE, \huge, \Huge
                    % if the current font size is 'normalsize', then using \relsize{-1} ==> go down one level ==> font = small
                    % and using \relsize{1}  ==> font = large
\newcommand{\myC}[1]{{\relsize{-1}{$\mathcal{#1}$}}} % as in 'my complexity class'

% padding in tables
\newcommand{\myPadTop}{6.5pt}
\newcommand{\myPadBottom}{4pt}

\mathchardef\myhyphen="2D
\usepackage{amssymb}%for QED symbol \blacksquare
%end_custom_header

%begin_custom_header
%end_custom_header
\begin{document}
%begin_custom_content
		\subsection{Proof}\label{sup_proof}

				%\noindent$\normalsize\boldsymbol{Proof:}$

				\vspace{5pt}
				\small{\noindent I. Define $\boldsymbol{\gamma}:\{1,..,r\}\rightarrow \{1,..,r\}$ \mys s.t. $\forall i, 1\leq i <r: w_{\gamma(i)} \leq w_{\gamma(i+1)}$

							\noindent II. Let $\boldsymbol{G}=\boldsymbol{O}+\{o_{r+1}\}$, \mys $\boldsymbol{t}=\boldsymbol{c}$,\mys $\boldsymbol{A}=(a_1,\dots,a_r, a_{r+1})$, \mys where \mys $a_{r+1}=0$ \mys and \mys $\forall i\leq r,$\mys$a_i=+1$

							\noindent III. Let $\boldsymbol  {M}$ \mys be a $d\times d$ zero-matrix, $d=r+1$. Populate $M$ as follows:

								\noindent\hspace{20pt}1. Repeat for \mys$i=1$\quad to\quad $i=r-1$:
							\setlength{\belowdisplayskip}{0pt} %\setlength{\belowdisplayshortskip}{0pt}\setlength{\abovedisplayskip}{0pt} \setlength{\abovedisplayshortskip}{0pt}
							%\begin{siderules}
							\begin{alignat*}{7}
										\qquad\qquad & j\mys       & \leftarrow\mys\mys   & \gamma(i)   & \mys\text{and} \quad\mys\mys & k\mys       & \leftarrow\mys\mys   & \gamma(i+1) \\
										\qquad\qquad & m_{jj}\mys  & \leftarrow\mys\mys   & v_j         & \mys\text{and} \quad\mys\mys & m_{jk}\mys  & \leftarrow\mys\mys   & -w_j \\
										\qquad\qquad & w_{k}\mys   & \leftarrow\mys\mys   & w_{k} - w_j & \mys                         &\mys         & \mys                 & \mys
							\end{alignat*}
                            %\end{siderules}
							\noindent\hspace{20pt}2. $j\leftarrow\gamma(r)$, $m_{jj}\leftarrow v_j$, $m_{dj}\leftarrow -w_j$

							\noindent IV. Calculate $\boldsymbol{B}$, $\boldsymbol{D}$ and define $\boldsymbol{f}:\boldsymbol{G}  \rightarrow \{0,1\}$ (Section \ref{sup_NEP_definition}).

							\noindent V. Return $(f(o_1), \dots, f(o_r))$ as KOP's solution vector \hspace{5pt}$\blacksquare$
						}

						\vspace{4pt}\noindent Proof notation follows that in KOP (above) and NEP (Section \ref{sup_NEP_definition}) definitions.

			%\begin{comment}
\subsection{Reverse-Reducing NEP To KOP}\label{reverse_reduction}
			While the KOP-to-NEP reduction proves the later to belong to the same complexity class as
			the former, NEP-to-KOP reduction allows the use of an existing well-known pseudo-polynomial dynamic-programming algorithm  \cite{pisinger_where_2005}
			to solve instances of the former. NEP can be reverse-reduced to KOP by setting $O=G, V=B, W=D$, and \textbf{$c=t$}.

			%\end{comment}
%end_custom_content
\printbibliography
\end{document}
