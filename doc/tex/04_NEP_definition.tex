\section{The Network Evolution Problem}
      The nodes in a MIN correspond to a set of genes $G=\{g_1,g_2,..,g_n\}$, and the directed and signed edges represent
       interactions. The direction of an edge between $g_j$ and $g_k$ denotes which of the two genes is the source and which is the target. The nodes in the hypothetical MIN shown Figure \ref{fig:intro_fancy} (A), left,  represent 7  genes, with promotional and inhibitory interactions denoted by arrow- and bar-terminated edges, respectively. A hypothetical `Oracle advice' (OA) on all or some of the genes simulates the evolutionary pressure on the network. The OA is represented as a ternary sequence $A = (a_1,a_2,\dots,a_n)$ where $a_j=+1$ or $a_j=-1$ indicates that the Oracle says $g_j$ should be promoted or inhibited, respectively. $a_j=0$ implies the Oracle is neutral towards $g_j$. In Figure \ref{fig:intro_fancy} (A) example, the OA is $A=(0,+1,-1,+1,0,0,-1)$, meaning the Oracle says genes $g_2,g_4$ should be promoted (notice $a_2=a_4=+1$) while $g_3,g_7$ should be inhibited ($a_3=a_7=-1$). The Oracle has no opinion towards genes $g_1,g_5$ and $g_6$ ($a_1=a_5=g_6=0$) in this example.

      The adjacency matrix $M$ in Figure \ref{fig:intro_fancy} (A), centre, is an equivalent representation of the network graph on the left. There is a non-zero entry $m_{jk}$ for each edge between genes $g_j$ and $g_k$. A promotional or inhibitory interaction is represented in $M$ as a $+1$ or $-1$ entry, respectively.  A green- or red-coloured matrix entry $m_{jk}$ denotes whether the interaction is in agreement or disagreement with the OA on $g_k$, i.e. $a_k$. While $m_{jk}$ describes what the effect of $g_j$ on $g_k$ actually \textit{is}, $a_k$ describes what that effect \textit{should} ideally be.
      %In other words, an entry  $m_{jk}$ is green if it has the same sign as $a_k$, and red otherwise. Interactions that are neither in agreement no disagreement with the OA are grey-coloured.

      For example, $g_1$ promotes  $g_2$ ($m_{12}=+1$) in agreement with the OA that $g_2$ should be promoted ($a_2=+1$). On the other hand, $g_1$ inhibits  $g_4$ ($m_{14}=-1$) in disagreement with the OA on $g_4$ ($a_4=+1$). The benefit/damage (b/d) score of a gene $g_i$ is calculated as the sum of green/red interactions along row $i$ (projection) and column $i$ (attraction), as shown in the right tabular in Figure \ref{fig:intro_fancy} (A). The benefit and damage scores of $g_3$, and the corresponding row and column from which they have been summed up, are highlighted with dotted frame in Figure \ref{fig:intro_fancy} (A). Notice that although the Oracle is neutral towards $g_1,g_5$ and $g_6$, they do have b/d values depending on their interactions with other genes and whether those interactions are in agreement with the OA or not.
      %\footnote{In this study we conducted simulations where the percentage of genes towards which the Oracle has an opinion varies from 100\% to to 25\% as will be detailed in the results section}.
      Given the set of genes with non-zero benefit and damage score, the optimization problem (formally defined in SI \ref{I-sup_NEP_definition}) is:

      \textit{What subset of genes should be conserved and which should be mutated/deleted so as to maximize  the total number of beneficial interactions network-wide, while minimizing (to a threshold) the total number of damaging interactions?}


      \begin{comment}
          An interaction $m_{jk}$ is beneficial if it is in agreement with what the Oracle says $g_k$ should be (i.e. either $(m_{jk}=+1$ AND $a_k=+1)$ OR $(m_{jk}=-1$ AND $a_k=-1)$), and damaging if it is in disagreement with what the Oracle says $g_k$ should be (i.e. either $(m_{jk}=+1$ AND $a_k=-1)$ OR $(m_{jk}=-1$ AND $a_k=+1)$). Each gene $g_j$ is henceforth assigned a benefit (damage) score $b_j$ ($d_j$) depending on how many beneficial (damaging) interactions it \textit{projects} onto or \textit{attracts} from other genes through its outgoing and incoming edges, respectively. Each beneficial (damaging) interaction therefore adds $|m_{jk}|$ to the benefit (damage) score of both the source gene $g_j$ and the target gene $g_k$. The rationale of accounting for both projection and attraction is to make the model indifferent to their distribution of the in- and out-degree of a gene.
      \end{comment}
