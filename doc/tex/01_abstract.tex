%\title{Evolution by Computational Selection\newline\newline}
%\title{The Effects of Computational Intractability Law on the Evolution of Biological Networks}
%\title{The Evolution of Biological Networks form a Computational Complexity Perspective}
%\title{The Molding of Biological Networks' Topology by Computational Intractability Law}
%\title{The Molding of the Topology of Biological Networks by Computational Intractability Law}


\begin{document}
{\centering {\Large Computational Intractability Law Molds the Topology of Biological Networks}}~\\~\\
\date{}

{\noindent\normalsize{\centering Ali A Atiia{\footnotesize$^{1,2}$},Corbin Hopper{\footnotesize$^{1,5}$}, Katsumi Inoue{\footnotesize$^{3,4}$}, Silvia Vidal{\footnotesize$^{2}$} \&  Jérôme Waldispühl{\footnotesize$^{1,*}$}}}

{\footnotesize
   \noindent$^1$\textit{School of Computer Science, McGill University, Montreal, Canada}~\\
   \noindent$^2$\textit{Research Centre on Complex Traits, McGill University, Montreal, Canada}~\\
   \noindent$^3$\textit{National Institute of Informatics, Tokyo, Japan}~\\
   \noindent$^4$\textit{Tokyo Institute of Technology, Tokyo, Japan}~\\
   \noindent$^5$\textit{École normale supérieure Paris-Saclay, Cachan, France}~\\
   \noindent$^*${Corresponding author: jeromew@cs.mcgill.ca}~\\
}


\vspace{25pt}
\noindent\textit{\textbf{Abstract:}} Virtually all molecular interaction networks (MINs), irrespective of organism or physiological context, have a majority of loosely-connected `leaf' genes interacting with at most 1-3 genes, and a minority of highly-connected `hub' genes interacting with at least 10 or more other genes.
    %
    Previous reports proposed adaptive and non-adaptive hypotheses describing sufficient but not necessary conditions for the origin of this majority-leaves minority-hubs (mLmH) topology.
    %
    We modeled the evolution of MINs as a computational optimization problem which describes the cost of conserving, deleting or mutating existing genes so as to maximize (minimize) the overall number of beneficial (damaging) interactions network-wide.
    %
    The model 1) provides sufficient and, assuming $\mathcal{P}\neq \mathcal{NP}$, necessary conditions for the emergence of mLmH as an adaptation to circumvent computational intractability, 2) predicts the percentage number of genes having $d$ interacting partners, and 3) when employed as a fitness function in an evolutionary algorithm, produces mLmH-possessing synthetic networks whose degree distributions match those of equal-size MINs.

  \vspace{33pt}
{\noindent\textit{\textbf{Author Summary}}: Our results indicate that the topology of molecular interaction networks is a selected-for adaptation that minimizes the evolutionary cost of re-wiring the network in response to an evolutionary pressure to conserve, delete or mutate existing genes and interactions.}
