\section{The Computational Complexity of NEP}
    The computational complexity of NEP is equivalent to that of the well-known \myC{NP}-complete knapsack optimization problem (KOP). Figure \ref{fig:intro_fancy} (B) shows a KOP instance with 7 items, each tagged with a certain value reflecting its utility, and some weight in pounds. The values and weights are indicated on items' tags with green and red numbers respectively. The optimization question in KOP is how to pack the knapsack with as many items as possible such that the total value of items included in the knapsack is maximal while their total weight does not exceed the knapsack maximum capacity, which is 5 pounds in this KOP instance. Some items have negligible weight and some are useless, such as pen and candle in this example, respectively. Clearly the pen (candle) should be included in (excluded from) the knapsack regardless of what other objects' fate, and as such need not be considered in the optimization search. Among the remaining objects, a search through the include/exclude combinations must be explored to determine the optimal solution.

    We say that the KOP instance in Figure \ref{fig:intro_fancy} (B) is reduced to the NEP instance in (A). Nodes $g_1, g_2, ..., g_7$ and their b/d scores in  (A) respectively correspond to the items and value/weight scores of the KOP items  (B). For example, KOP's laptop corresponds to NEP's gene $g_2$. The optimal solution to the NEP instance in Figure \ref{fig:intro_fancy} (A) would be to conserve $g_2, g_4,  g_5$ and  $g_6$ and to mutate or delete $g_1, g_3$ and $g_7$, assuming the maximum threshold of tolerable damaging interactions to be 5 (corresponding to the 5 pounds knapsack capacity in (B)). This solutions respectively translate back to an optimal solution to the KOP instance in (B):  include in  the knapsack the laptop, lunch box, pen and water bottle, and exclude the notebook, textbook, and candle. Generally, any instance of KOP can easily (i.e. with polynomially bounded computational cost) be reduced into a corresponding  NEP instance. A generalized formal KOP-to-NEP reduction is included in SI \ref{I-KOPdefinition}.

    With small number of KOP items it is trivial to find out the optimal solution, but as the number of items increases the combinatorial search space grows exponentially fast. Whether there exists a polynomially-bounded algorithm for solving any arbitrary instances of an \myC{NP}-complete (\myC{NPC}) problems is the subject of the \myC{P} vs. \myC{NP} question, arguably the most important questions in computer science and mathematics today \cite{aaronson_limits_2004, aaronson_guest_2005, fortnow_status_2009}. \myC{NPC} problems have defied all attempts aimed at finding algorithms that consume polynomially-bounded amount of computational resources for solving arbitrary instances. Hence, the increasingly accepted conjecture is that problems in this class will always require super-polynomial computational resources, and that this should be accepted as a universal law by the same token that repeatedly experimentally verified laws in physics are accepted as such \cite{wigderson_opening_2014}.
    %
    In practice heuristics-based algorithms do exist and can find good approximation solutions (e.g. by exploiting structures in practically common instances) \cite{vazirani_approximation_2013,lawler_fast_1979}. In evolutionary context, however, the only `algorithm' at hand is random variation non-random selection (RVnRS), and the NEP instance difficulty is measured relative to it.
