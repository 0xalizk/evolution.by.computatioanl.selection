\subsection{simulated evolution and adaptation:}\label{evolution_results}
    We conducted simulated network evolution and adaptation using an evolutionary algorithm that selects for networks that best minimize the difficulty of NEP instances generated against them. The simulation aims to assess the potency of NEP as a selection pressure that shapes network topology, rather than to mirror the details of recombination and mutation processes in biological systems. We conducted two sets of simulations depending on the starting conditions of synthetic networks. In the first, we begin with empty networks that grow over the generation by accumulating more nodes and edges. In the second, we begin by synthetic networks that have the same size (number of nodes and edges) as corresponding real MINs, which is kept unchanged from one generation to the next, but with an edges getting re-assigned randomly in each generation as a form of mutation. We refer to the first and second sets of simulations as 'evolution' and 'adaptation' simulations respectively. In both experiments, the idea is to examine the degree distribution of synthetic networks after iterations of random mutation non-random selection, and compare that  to the degree distribution of biological networks of equal size.

    In simulated evolution, networks at the beginning of the simulation are near empty but are periodically assigned more new nodes and edges, in addition to being mutated by randomly re-assigning an existing edge to two randomly selected nodes. The simulation begins with a population of 60 near empty synthetic networks (8 nodes and $8r$ randomly assigned edges, where $r$ is the edge:node ($e2n$) ratio to be maintained throughout the simulation). An NEP instance against each network in the population is obtained by generating a random interaction-targeting OA, resulting in some interactions in the network being beneficial and others damaging. Nodes' benefit/damage scores are calculated (as illustrated previously by example in Figure \ref{fig:intro_fancy}, and further detailed in SI \ref{I-matrix_OA}). Assuming a tolerance threshold of 5\% of total damaging interactions network-wide, instances are solved to optimality (detailed further in SI \ref{I-sup_algorithmic_workflow}).

    The fitness of each synthetic network is based on two values averaged over the 60 instances: 1) effective instance size (EIS) as described in Section \ref{sec:EIS}  and 2) the effective gained benefits (EGB) which measures the total number of beneficial interactions than can be obtained with the least number of nodes having to be conserved/deleted in an optimal NEP instances solution (see SI \ref{I-sup_sim_evo} for details). The top 6  fittest  networks (10\% of the population) that best minimize EIS and maximize EGB survive, while the remaining die. The surviving 6 networks are each mutated (add-node, add-edge, and re-assign edge) and 10 exact replicas are bred from each for the next round of mutate-and-select, resulting in a population of 60 networks once again. Another round starts by generating 100 NEP instances against each of the 60 networks in the population, and so on. The simulation is terminated when the number of nodes reaches that of a corresponding real MINs. Throughout the simulation, the add-edge mutation is conducted only if the edge-node ratio ($e2n$) of the synthetic network does not exceed that of the corresponding real MIN.

    Figure \ref{evol_figure} illustrates a sample of evolved synthetic networks and the corresponding MINs that have the same $e2n$ ratio (results for all other networks are shown in SI \ref{I-sup_sim_evo}). The degree distributions of simulated networks (blue in Figure \ref{evol_figure}) closely match their corresponding MINs (pink). In simulated adaptations, the starting synthetic networks have the same size as the corresponding MIN, and no add-node or add-edge mutations are conducted (i.e. synthetic networks don't grow in size from one generation to the next). Only reassign-edge mutation is conducted in each generation. All other aspects of the mutation/selection process is the same as in the aforementioned simulated evolution experiments. mLmH property still emerges (see plots in SI \ref{I-sup_sim_adapt}). These results show the sufficiency of NEP-based evolutionary pressure to mold the network to mLmH topology within a number of generations that is extremely small (proportional to the number of genes in the target real MIN, i.e. ${\sim}$thousands, see SI \ref{I-sup_realnets} for network sizes) relative to real evolutionary time. These results also confirm findings from previous simulation and adaptation experiments on a smaller set of networks reported in \cite{atiia_computational_2017-1}.
