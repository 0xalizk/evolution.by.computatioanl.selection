\subsection{gene neutrality:}\label{neutrality}
To test the affect of having a certain fraction of the genes as neutral under a given evolutionary pressure, NEP instances were generating assuming 0, 25, 50, and 75\% of genes are neutral (the Oracle has no opinion about them). A gene towards which the Oracle is indifferent may nonetheless end up having a non-zero benefit and/or damage value depending on its interaction profile with other genes on which the Oracle has an opinion. Subplots from top-left (0\%) to bottom-right (75\%) in  Figure \ref{fig:neutrality} (A) demonstrate the resulting effective instance size (EIS) as pressure decreases (i.e. as the number of neutral genes increases). The bar height represents the effective instance size (EIS), i.e. the fraction of genes that are ambiguous (in that they are engaged in both beneficial and damaging interactions in a given NEP instance). EIS shown here is the average over 1-5K instances. As the pressure increases (bottom to top), the supremacy of real MINs compared to synthetic analogs (no-hubs (NH), no-leaves(NL) and random (RN) networks) is more pronounced. Leaf-deprived NL synthetic analogs show the largest EIS across all settings, while NH and RN networks show decreasing EIS to the extent that they both have more smaller-degree leaf nodes. The constituent nodes in each $b$:$d$ bar segments is shown in Figure  \ref{fig:neutrality} (B) for the  Fly network and its corresponding random analog. As the pressure decreases from 0\% to 75\%, the constituent nodes contributing to a given $b$:$d$ ratio range remains largely unchanged as the nodes towards which the Oracle is not indifferent are selected randomly and as such the relative frequency of nodes of a certain degree remains of the same proportion under different neutrality assumptions.
