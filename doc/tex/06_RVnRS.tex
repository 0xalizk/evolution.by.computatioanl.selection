\section{Optimization in Biological Context}\label{opt_in_bio_context}
    Biological systems do not employ sophisticated search algorithms to determine the optimal conserve or mutate/delete actions from one generation to the next, but rather proceed thru iterations of RVnRS \cite{carvunis_proto-genes_2012}. However, the number of needed RVnRS iterations before the composition (nodes) and connectivity (edges) of a network has sufficiently been transformed away from a deleterious state depends directly on network topology.
    %
    Particularly, the number of RVnRS iterations is exponential in the number of ambiguous genes (those having non-zero score for both benefit and damage).
    %Corbin: "I feel like this needs one sentence before it explaining why unambiguous genes would requires less mutations, ect to resolve. it may help to emphasize that contradictory needs (or neighboring genes) exponentially increases the number of RVnRS rounds needed
    %
    The more ambiguous a set of genes are, the less likely it is for the iterative RVnRS process to \textit{alter just the right set of genes} such that the total number of beneficial/damaging interactions is optimally maximized/minimized in as few evolutionary iterations as possible.

    The \myC{NP}-hardness of NEP in and of itself is a rather weak measure of computational difficulty as there are in practice heuristics-based approximation algorithms that can produce fairly satisfactory solutions that may not be too suboptimal relative to \textit{the} optimal solution.
    However, the efficacy of RVnRS relies on accumulating improvements with as little `backtracking` through the search space as possible. The more unambiguously beneficial or damaging genes there are in a system that is under some evolutionary pressure to alter its MIN, the more effective RVnRS is at accumulating more beneficial, and cleansing more damaging, interactions.

    A naive greedy strategy could be to impose the OA by conserving every gene $\boldsymbol{g_i}$ where $a_i = +1$, and mutating/deleting every $g_j$ where $a_j=-1$. However, \textbf{conserving} $\boldsymbol{g_i}$ can inadvertently conserve OA-contradicting interactions if $g_j$ happens to be a promoter or inhibitor of some $g_k$ where $a_k=-1$ or $a_k=+1$ respectively. And \textbf{deleting/mutating} $\boldsymbol{g_j}$ can inadvertently disrupt OA-supporting interactions if $g_j$ happens to be a promoter or inhibitor of some $g_k$ where $a_k=+1$ or $a_k=-1$ respectively. In other words, a naive imposition of an OA is complicated by the reality of network connectivity. NEP optimization in biological evolutionary context is summarized in Table SI \ref{I-informal_table}. Further details on the notions of gene ambiguity, search space size under RVnRS, and the role of mLmH topology at reducing it in Subsection \ref{sec:EIS} and \ref{sec:search_space}.
