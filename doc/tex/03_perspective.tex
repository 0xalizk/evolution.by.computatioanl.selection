\section{Perspective}
        A fundamental property in MINs is the majority-leaves minority-hubs  (mLmH) topology whereby an overwhelming ${\sim}$80\% majority of leaf genes interact with at most 1-3 other genes, and an elite ${\sim}$6\% minority of hub genes interact with at least 10 other genes (these figures are the average over the 25 networks featured in this article which, to our knowledge, represent all large-scale experimentally validated networks of direct physical interactions available in the literature). mLmH-possessing networks are typically referred to as 'scale-free' (SF). Because of the association between the SF term and the controversial proposition \cite{arita_metabolic_2004, tanaka_protein_2005, fox_keller_revisiting_2005, khanin_how_2006} that biological networks follow a power-law distribution, we use the loose term mLmH to emphasize the fact that our model does not assume, require nor advocate for or against the idea that the degree distribution of MINs follows a power-law in a strict technical sense.
        mLmH is observed in virtually all MINs irrespective of organism or physiological context as shown in Figure \ref{mLmH_fig}. Here we investigate the mLmH topology from a computational complexity perspective and derive necessary and sufficient conditions for its emergence. We assume that the organism is under some evolutionary pressure to change. Under this hypothetical scenario, we assume it critical for the system, in some regulatory state at some particular point in evolutionary time,  that some genes be promoted and some be inhibited. An interaction is deemed beneficial if its promotional or inhibitory towards a gene that should indeed be promoted or inhibited respectively. Similarly, an interaction is deemed  damaging if it is promotional or inhibitory towards a gene that should rather be inhibited or promoted, respectively. We assign a benefit (damage) score for each gene $g_i$ as the sum of beneficial (damaging) interactions that $g_i$ is \textit{projecting onto} or \textit{attracting  from} other genes in the network. If gene $g_i$ promotes or inhibits $g_j$, we say that $g_i$ is projecting an interaction onto $g_j$, and $g_j$ is attracting an interaction from $g_i$. The benefit or damage scores of both $g_i$ and $g_j$ is incremented by some some $\rho\in\mathbb{R}$ If such an interaction is beneficial or damaging, respectively. $\rho$ signifies the potency of such an interaction. A beneficial/damaging interaction therefore adds $\rho$ to the total benefit or damage score of both the source and target genes. A gene may for example be projecting very few beneficial interactions while attracting many damaging interactions (making it more of a liability) and vice versa. Hence the benefit/damage scoring is not affected by how skewed the in- versus out-degree distribution of a gene is. Because we have no knowledge of what $\rho$ is (i.e. measure of interaction potency) at a global scale in MINs, we use $\rho=1$ for all interactions in this study.

        Given the benefit and damage scores of genes under the current evolutionary pressure scenario, how hard of a computational problem would it be to determine the optimal immediate "next-move" for the system, i.e. which genes to conserve, mutate or delete, such that the overall total number of beneficial interactions is maximal \textit{while} the total number of damaging interactions is minimal to a threshold?
        %In evolutionary context, we characterize hardness by the necessary number of random-varation non-random selection iterations before the network's composition (genes) and connectivity (interactions) have been transformed to satisfy this maximization/minimization target.
        We refer to this optimization question as the network evolution problem (NEP).
