\section{The Semantics of NEP}\label{sup_semantics}
      The OAs described thus far represent rather blunt opinions about a gene by stating it should be promoted or inhibited. In reality a gene may ideally have both a promoter and an inhibitor co-expressed whereby, for example, the latter serves to tone down the potency of the former. A more fine-tuned OA would therefore be on individual interactions rather than genes. The only syntactic modifications to NEP definition in this case would be that the OA is a matrix that is equal in dimensions to the interaction matrix $M$. This is in contrast to a gene-targeting OA which is represented as a $|G|$-long sequence where $|G|$ is the total number of genes in the network (as was the case in the example of Figure \ref{fig:intro_fancy} A). The complexity and instance difficulty of NEP remains the same whether the Oracle is stating its advice on genes or interactions, however (see SI \ref{I-matrix_OA} for further details).
      

			An OA can further be interpreted semantically as describing short- or long-term evolutionary pressure. Short-term OA is specific to some regulatory state at a specific moment of the cell's life cycle for example (Figure \ref{fig:intro_fancy} (C)). An OA on some gene or interaction indicates an evolutionary pressure to fine-tune its regulation positively or negatively. Long-term OA on the other hand indicates a more existential evolutionary pressure for or a against a gene or interaction. In this case, the OA is interpreted as advising on whether a gene or interaction is rather advantageous or disadvantage to the organism's survival generally (Figure \ref{fig:intro_fancy} (D)).
