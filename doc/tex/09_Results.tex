\section{Results}
    Here we summarize the results presented in subsequent subsections. We empirically study NEP instances obtained by generating hypothetical OAs on various MINs and compare that with instances on synthetic networks of various topologies. We assess instance difficulty assuming the working algorithm is RVnRS. The results show instances obtained from  real MINs are invariably far easier to satisfy by virtue of the mLmH property, as real MINs have far less ambiguous genes per instance. The large number of low-degree leaf genes in MINs reduces  instance size because such genes are certain (degree 1) or likely (degree 2, 3, .. with exponentially decreasing likelihood) to have all-beneficial  (all-damaging)interactions and should therefore be conserved (mutated/deleted) regardless. Such unambiguous genes need not be considered in the computationally costly optimization search. For example, a leaf gene involved in only one interaction (i.e. it has degree 1)  can either be totally beneficial or totally damaging  under a given evolutionary pressure scenario, and as such there is no ambiguity as to whether it should be conserved or mutated/deleted. A gene of degree 2 has a 50\% chance of being unambiguous: the two interactions it is engaged in are either both beneficial or both damaging (a degree-2 gene is engaged in two interactions. There are four possibilities of the state of the two interactions are (0,0), (0,1), (1,0) and (1,1) where 1 or 0 denote an interaction is beneficial or damaging respectively).  In general, a gene of degree $d$ has a $\frac{2}{2^d} = 2^{d-1}$ probability of being unambiguous.

    %Furthermore, as leaves minimally consume the tolerance threshold,  more hubs can be conserved for their benefits and despite their damages.  A hub gene can single-handedly contribute a large portion of the total beneficial interactions, allowing for  the packing of more beneficial interactions with less genes to conserve/delete.

    Based on the fact that the larger a gene's degree is the exponentially more likely it is to be ambiguous,  the model predicts the expected number of genes of degree $d$ in real MINs. The prediction formula is parameterized with a value proportional to the  edge:node ratio of the MIN. All existing MINs are currently partial to different degrees of completion (i.e. not all interactions of all genes have been mapped out). Based on the predictability of the degree distribution of MINs from a computational intractability perspective, we conjecture that, once all genes and interactions have been accounted for, there will be a universal  edge:node ratio of ${\sim}$2 in all MINs regardless of organism or physiological context.


    We also simulated the evolution of synthetic networks using an evolutionary algorithm which, in each generation, selects the top 10\% of networks that present the easiest optimization task (mainly by having a large number of unambiguous nodes). These top networks  breed the next generation of networks. The degree distribution of the fittest synthetic networks after  ${\sim}$hundreds - few thousands of generations  of simulated evolution very closely match those of real MINs of equal size.  The emergence of mLmH in synthetic networks is not sensitive to the starting conditions: whether networks are initially empty (and accumulate nodes/edges over the generations) or randomized (have the same number of nodes/edges as a corresponding real MIN, but edges are re-assigned over the generations as a form of mutation).

    \begin{comment}
        In light of the ongoing debate around the evolutionary advantage of structural properties of MINs and the universal laws that have shaped their evolution,  the presented results indicate the mLmH property minimizes the computational costs (the number of RVnRS) of rewiring the interaction network in response to  an evolutionary pressure to change. The presented model provides sufficient conditions for predicting and evolving mLmH-possessing synthetic networks whose degree distribution closely match real MINs of equal size. The model provides a necessary condition in that deviation from the mLmH topology necessarily (assuming \myC{P}$\neq$\myC{NP}) leads to an exponential increase in the search space that RVnRS must explore before the network has sufficiently been transformed away from a deleterious and into an advantageous state.
    \end{comment}
