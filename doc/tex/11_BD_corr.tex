	\subsection{benefit-damage correlation:}
		The correlation between benefit and damage scores is used to assess difficulty of NEP instances analogously to values-weights correlation in KOP (see NEP-to-KOP reverse-reduction in SI \ref{I-reverse_reduction}). Figure \ref{scatter} (A) shows the correlation plot of classical test instances \cite{pisinger_core_1999-1}. It has previously been shown that the more correlated the values and weights in a knapsack instance are the more difficult the instance is \cite{pisinger_where_2005}. Strong value-weight correlation increases the ambiguity as to which items to add/remove from the knapsack (or in NEP context, which genes to conserve/delete from the network). Figure \ref{scatter} (B) shows the average frequency of a (benefit, damage) pair ($(b,d)$ hereafter) over 1-5K NEP instances for each network. The reduced ambiguity in instances from real PPI networks results by virtue of the large number of genes that are certainly (degree 1) or likely (degree 2, 3, 4 .. with likelihoods 50, 12.5, 0.125 .. \%, respectively) to be unambiguous: either totally advantageous ($b\neq 0$, $d=0$) or totally disadvantageous ($b=0$, $d\neq 0$). In Fly PPI network for example, ${\sim}50\%$ of all $(b,d)$ pairs are unambiguous,  with 1:0 and 0:1 pairs (resulting from degree-1 leaf genes) alone representing  ${\sim}43\%$  of those pairs (large brown dots). In contrast, ${\sim}15.6, {\sim}3.1,$ and ${\sim}28 \%$ of $(b,d)$ pairs in Fly network's  NH, NL, and RN analogs are unambiguous. The role of leaves in decreasing $(b,d)$ correlation is especially highlighted in contrast to leaf-deprived NL network which exhibits the strongest correlation around its higher mean degree (analysis of degree-to-ambiguity proportional relation is detailed in "Prediction of degree distribution" section). The symmetry along the diagonal in Figure \ref{scatter} (B) is expected given that there is an equal probability of $50\%$ that an interaction is beneficial or damaging under a random Oracle advice on the gene that interaction is targeting.

		The signed Fly PPI network distinctly shows more sporadic benefit-damage correlation (red dots in Fly dot plot in Figure \ref{scatter}) in higher-degree nodes, which results from the asymmetry of the number of promotional (67.5 \%) and inhibitory (32.5 \%) interactions. Under random OA, such asymmetry results in higher likelihood of disparate benefit and damage values. The asymmetry of signs in Fly PPI network is consistent with a recent report that showed a similar promotional/inhibitory interaction distribution in yeast \cite{costanzo_global_2016}. The benefit-damage correlation for regulatory and DB-sourced networks manifest the same property (see SI \ref{I-sup_extra_BD_corr}).
