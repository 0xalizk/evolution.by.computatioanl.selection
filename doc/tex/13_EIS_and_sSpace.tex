	\subsection{effective instance size:}\label{sec:EIS}
			%The effective instance size is the fraction of genes having both non-zero benefit and damage.
			Unambiguous genes can \textit{a priori} be deemed  advantageous or disadvantageous and therefore should be conserved or mutated/deleted, respectively.  As such they need not be part of the optimization search because the optimal evolutionary outcome as to whether to conserve or alter them is independent of that of other genes in the network. Effective instance size (EIS) is the fraction of genes in an NEP instance that are ambiguous ($b\neq 0$ and $b\neq 0$). The smaller the $|b-d|$ value the more ambiguous a gene is. Figure \ref{wheel} (A), left (pie charts) show the  fraction of genes that on average (over 1-5K instances) falls under a certain $b$:$d$ ratio slice where $b$:$d$ =  $\frac{b}{b+d}$:$\frac{d}{b+d}$ for an example network (Fly PPI). NEP instances in PPI have ${\sim}42\%$ EIS ($b$:$d$ = 90:10, 80:20, .., 10:90\%), compared to ${\sim}84$, ${\sim}100$, and ${\sim}72$ \% for NH, NL and RN networks respectively. Compared to NH and NL, EIS in RN is smaller to the extend that it has more leaf nodes (particularly degree 1-3) which relatively increase the size of its unambiguous slices ($b$:$d$ 100:0 or 0:100\%).

      The constituent genes in each pie slice are shown in Figure \ref{wheel} (A), right bar charts, for each network, broken down by degree range (bottom legend). Since the likelihood of a gene's ambiguity is inversely (and exponentially, see later discussion in the next section) proportional to its degree, leaf genes (degree $\leq 4$) in PPI dominate the unambiguous 100:0 or 0:100\% $b$:$d$ ratio groups. With virtually all nodes  in NH network having a degree 4 (NH bar chart in Figure \ref{deg_dist} (A)), no $b$:$d$ ratio of any node can fall in certain $b$:$d$ ratio groups (more specifically, none of the possible ($b$,$d$) pairs 4:0, 3:1, 2:2, 1:3 ... 0:4 falls into any of the $b$:$d$ ratios 90:10, 80:20, 10:90, or  20:80 \%). Figure \ref{wheel} (B) shows the EIS for all networks, where the bar height represents the size of ambiguous pie slices shown in (A) excluding the 100:0 and 0:100 slices. Each bar group corresponds to a network, with the first bar being the real network and the 2nd, 3rd and 4th (left to right) corresponding to the network's NH, NL and RN analogs respectively. EIS is significantly smaller in real networks as compared to synthetic analogs regardless of physiological context or the source of the network. EIS in RN analogs is comparatively smaller in networks having smaller edge:node ($e2n$) ratio since such networks are more likely to have leaf nodes after random re-assignment of edges (e.g. Human and Yeast PPI networks have 1.45 and 3.24 $e2n$ ratios respectively).

	\subsection{search space size:}\label{sec:search_space}

			The total search space of a given NEP instance is defined as $S_t = 2^{O(n)}$ where $n$ is the number of relevant genes  in the instance (those with $b\neq 0$ and $d\neq 0$). The base 2 denotes the two general evolutionary outcome on a relevant gene $g_i$: its state will have been (1) altered or (2) unaltered after the next round of RVnRS. The alteration to a gene (for example through mutation) can be (dis-)advantageous relative to the current NEP instance. An advantageous change includes for example the deletion of a unambiguously damaging gene. The state of an ambiguous gene can further be affected by the state changes of its interacting partners. For example, in the NEP instance of Figure \ref{fig:intro_fancy} (B), a mutation to gene $g_2$ which causes it to lose its positive interactions with say,  $g_1$ and $g_5$ would not only change its benefit/damage $(b,d)$ scores from $(4,2)$ to $(2,2)$  but it would also result in the $(b,d)$ scores of $g_1$ and $g_5$ to change even if they themselves did not undergo any alteration. $g_1$ and $g_5$ lose 1 benefit as a result of them no longer positively interacting with  $g_2$ due to the  latter's aforementioned mutation.

			Clearly the more interconnected a gene is the more its state alteration, and the alteration of its interacting partners, changes the optimal evolutionary target for such a gene from one generation to the next (i.e. whether it should ideally be conserved or mutated/deleted given the current evolutionary pressure described by the OA). Effective search space is  defined as $S_e=2^{O(m)}$ where $m\leq n=\sum\limits_{i=1}^{n}{1-\delta_i}$ and  ${\delta}_i=\frac{|b_i-d_i|}{b_i+d_i}$ is the ambiguity measure of $g_i$ under the current NEP instance. On the one extreme where $\forall g_i$, $b_i=d_i\neq 0$, then $m=n$ and therefore $S_e=S_t$. On the other extreme where all genes are unambiguous, $\forall g_i$, $b_i=0|d_i=0$, then $S_e=0$.  An unambiguously beneficial or damaging gene $g_j$  ($d_j=0$ or $b_j=0$, respectively) does not increase $m$ regardless of the current evolutionary pressure scenario because $\delta_j = 1$   (gene ambiguity vs.  degree is discussed in details in the subsequent section "Prediction of degree distribution"). Figure \ref{fig:search_space} shows $S_t$ (grey) and $S_e$ (blue) for all networks. Note that the $y-$axis values are logarithmic in base 2, and hence linear differences between bar heights imply exponential differences in $S_t$ or $S_e$. The smaller but denser NL networks show smaller $S_t$ but suffer from exponentially higher $S_e$:$S_t$ ratio compared to other networks. This implies that, while the search space is smaller, the optimal solution can change radically after each RVnRS given how deeply interconnected the genes are in such a network. Biological networks show exponentially smaller $S_e$:$S_t$ ratio especially for larger more complete  (e.g. ENCODE and Human  PPI connectome) and/or curated (e.g. TRRUST) networks.
