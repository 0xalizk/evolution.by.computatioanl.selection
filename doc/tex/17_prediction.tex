	\subsection{prediction of degree distribution:}\label{prediction_section}
			We considered whether computational intractability alone can predict the degree distribution of a biological network. More precisely, we considered whether the likelihood of a gene of degree $d$ ("degree-$d$ gene" hereafter) to be totally advantageous or disadvantageous (belonging to green or red pie slices in Figure \ref{wheel} (A), respectively), which is exponentially inversely proportional to its degree, can predict the expected number of degree-$d$ genes in a biological network.
			For a degree-2 gene $g_i$, for example, there are $2^2=4$ potential states of benefits/damages that $g_i$ can assume under a given OA: 00, 01, 10, or 11 where 0 or 1 signify the edge (interaction) as being  beneficial or damaging, respectively. States 01 or 10  are "ambiguous": $g_i$ must be part of the overall optimization search to determine whether to conserve or delete it. In general, the number of ambiguous states for degree-$d$  gene is $2^d - 2$, albeit not all of equal ambiguity: while the 1000010 and 1111000 states of a degree-7 gene are both ambiguous, the former is significantly less so. Let $k$ correspond to the number of 1's in a gene's given state (equivalently, its benefit score in a given NEP instance). We refer to a given state of a degree$\myhyphen d$ gene as $k\myhyphen ambiguous$ ($k\myhyphen amb$ hereafter), $0 \leq k \leq d$, if it has $k$ 1's. For example, 0111 and 1000 are  $3\myhyphen amb$ and $1\myhyphen amb$ states of a degree-4 gene. As $k\to d$ (or $k\to 0$) the ambiguity whether to conserve (or delete) decreases, while as $k\to \frac{d}{2}$ both the ambiguity and (exponentially) number of states increases.
			%The larger the degree, the greater the number of its ambiguous states, with those most ambiguous ($k$)being exponentially more likely to arise.
			For a degree-20 gene for example, there are ${20 \choose 3} = 1140$ $3\myhyphen amb$ states compared to ${20 \choose 10}=184756$ $10\myhyphen amb$ states. Assuming an equal probability $q=\frac{1}{2}$ for an edge to be beneficial or damaging, the likelihood of $k\myhyphen amb$ state for degree-$d$ gene is given by the expected number of $k$ successes in $d$ Bernoulli trials:
			%\setlength{\belowdisplayskip}{0pt} \setlength{\belowdisplayshortskip}{0pt}\setlength{\abovedisplayskip}{0pt} \setlength{\abovedisplayshortskip}{0pt}

            \begin{equation*}
						P(k\myhyphen amb) = {d \choose k}q^d = {d \choose k}2^{-d}
			\end{equation*}

			\noindent We define the expected frequency of degree-$d$ genes as:
			%\setlength{\belowdisplayskip}{0pt} \setlength{\belowdisplayshortskip}{0pt}\setlength{\abovedisplayskip}{0pt} \setlength{\abovedisplayshortskip}{0pt}
            \begin{equation*}
						E(d)= \frac {\alpha}{d+1}\sum\limits_{k=0}^{d} P(k\myhyphen amb)^{\beta log(d)}
			\end{equation*}

            \noindent where constants $\alpha, \beta \in \mathbb{R^+}$ are proportional to node:edge (n2e), edge:node ($e2n$) ratios, respectively. Figure \ref{predict_figure} (A) shows the actual (pink) and predicted (green)  degree frequency in Fly PPI network   at $\alpha, \beta = 0.43, 1.9$, respectively. Prediction is further applied  to all other networks with   accuracy (defined as 100 - $\sum |predicted(d)-actual(d)|$) being $>=84\%$ as shown in Figure \ref{predict_figure} (B). Individual prediction plots for all networks is included in SI \ref{I-sup_prediction}.  An accurate account of all interactions currently is affected by the experimental bias against interactions involving lesser known genes (an inherent problem to small-scale studies \cite{rolland_proteome-scale_2014}). An accurate account of all genes is not only limited by experimental coverage but also  by alternatively-spliced isoforms of the same gene (which can have distinct interaction profiles \cite{yang_widespread_2016}) being treated as a single gene, hence a gene's degree may be inflated in experiments where isoforms are not distinguished. For example, nodes in HumanIso network can be different isoforms of the same gene \cite{yang_widespread_2016}.
            %${\sim}27\%$ of nodes in HumanReg network are transcription factors (TF), while the rest are non-TF genes \cite{han_trrust:_2015}.
            The  ($\alpha$, $\beta$) values versus ($n2e$, $e2n$) ratios in a partial MIN should indicate how well its coverage and resolution compares to other standard high-quality MINs. Figure \ref{predict_figure} (C) shows ($\alpha$, $\beta$) versus  ($n2e$, $e2n$) ratios of networks in (B) (all of which are currently partial, see SI \ref{I-sup_realnets}). The average  of ($\alpha$ vs. n2e), ($\beta$ vs. $e2n$) are (0.43$\pm$0.063 vs. 0.526$\pm$0.133), (1.96$\pm$0.324 vs. 2.06$\pm$0.634) respectively. We conjecture that there is ultimately (as experimental coverage and resolution of high-throughput interaction-detecting experiments increases) a universal $e2n$ ratio of ${\sim}$2 in all MINs .
