\subsection{simulation of evolutionary pressure:}
      NEP instances were generated on MINs (previously described in Figure \ref{mLmH_fig}, see also SI \ref{I-sup_realnets} for extensive details) as well as various synthetic networks. For each MIN, we computer-generated synthetic analog networks having the same number of node and/or edges. Figure \ref{deg_dist} (A) shows the degree distribution of an example real network (\textit{Drosophila melanogaster} (Fly) protein-protein interaction (PPI) network \cite{vinayagam_integrating_2014}).
      %Although there are numerous large network datasets available,
      %very few have directed and signed edges (i.e. the nature of the interactions, whether inhibitory or promotional in nature, is unknown).
      %Directed-signed interaction networks available in KEGG for example are small in size.
      %The ER network is computer-generated \cite{schult_exploring_2008} by adding an edge with equal probability
      %between two nodes, and so the nodes' degrees cluster around the (\textit{no. edges}) $\div$ (\textit{no. nodes}).
      %This is equivalent to the
      %re-assignment of each edge in PPI to two randomly selected nodes.
      A no-leaves (NL) and no-hubs (NH) networks were generated by reassigning edges in PPI from leaves to nodes (for NL) or vice versa (for NH),
      so as to simulate the effect of depriving PPI of either property.
      Nodes in NL (NH) have a minimum (maximum) degree  $\geq$ ($\leq$) the average
      node degree of PPI ($\ceil{{\sim}3.6}=4$). The redistribution of edges from leaves to hubs in NL results in some nodes having zero degree, which are eliminated,
      resulting in NL being a smaller (943 nodes), more dense network compared to PPI.
      NL and NH networks
      simulate two alternative topologies that biological networks could have evolved into if minimizing interactions per gene (NH) or total number of genes (NL)
      were the only driving forces in their evolution.
      We also applied the same simulation to a random (RN)  analog of PPI network, whereby each edge in the latter is re-assigned to two randomly selected nodes
      (both edge direction and sign
      randomly assigned), and so nodes' degrees cluster around the average degree in PPI network.
      %RN is almost equivalent to an Erdos-Renyi network (also referred to as exponential network) generated using probability = $\frac {E}{N^2}$ where $E,N$ are the number of
      %edges, nodes in PPI.
      Figure \ref{deg_dist}  (B)-(D) respectively show the degree distribution of analog (NL, NH, and RN) random networks for protein-protein, regulatory, and database-sourced networks (the degree distribution of corresponding real networks were shown in Figure \ref{mLmH_fig}).
      1-5K NEP instances are generated for each network by calculating $B$ and $D$ values against a randomly generated OA on all nodes (for each node $n_i, a_i\neq 0$).
      Each instance is solved to optimality under a given tolerance threshold, expressed as the \% of damaging edges to be tolerated. Details of
      the algorithmic workflow of the simulation, and the choice of sampling threshold, is provided in SI \ref{I-sup_algorithmic_workflow}, \ref{I-sup_1Kvs5K}, respectively.
